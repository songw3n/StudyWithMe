\documentclass{article}

\usepackage[margin=1in]{geometry}
\usepackage{enumitem}

\setlength{\headsep}{0.5 in}
\setlength{\parindent}{0 in}
\setlength{\parskip}{0.1 in}
\addtolength\topmargin{-5mm}
\textheight  9.5in

\usepackage{amsmath,amsfonts,graphicx,multicol, color}
\usepackage{ulem}

\newcommand{\todo}{\textcolor{red}}
\newcommand{\red}{\textcolor{red}}
\newcommand{\eat}[1]{#1}

\usepackage{url}

%\def\IsSolution{1}   % Uncomment to debug solutions. Note: Makefile generates both discussion and solution automatically.

%
% Define two macros
%   \soln{}     --- contents appear only in solution
%   \notsoln{}  --- contents appear only if not solution

\usepackage{etoolbox}
\ifx\IsSolution\undefined
 \newcommand{\notsoln}[1]{#1}
 \newcommand{\soln}[1]{}
 \newcommand{\thesoln}[1]{}
\else
  \newcommand{\notsoln}[1]{}
  \newcommand{\soln}[1]{#1}
  \newcommand{\thesoln}[1]{{\color{red} \textbf{Solution:} #1}}
\fi


%
% The following macro is used to generate the header.
%
\newcommand{\header}[1]{
   \pagestyle{myheadings}
   \thispagestyle{plain}
   \newpage
   \setcounter{page}{1}
   \noindent
   \begin{center}
   \framebox{
      \vbox{\vspace{2mm}
    \hbox to 6.28in { {\bf COMPSCI 311: Introduction to Algorithms
		\hfill Spring 2019} }
       \vspace{4mm}
       \hbox to 6.28in { {\Large \hfill Homework #1 \soln{Solutions} \hfill } }
       \vspace{2mm}
       \hbox to 6.28in { Released 1/24/2019 \hfill Due \sout{2/7/2019} 2/8/2019 11:59pm in Gradescope}
      \vspace{2mm}}
   }
   \end{center}
   \markboth{Homework #1}{Homework #1}

   \vspace*{4mm}
}

\newcounter{problem}
\newcommand{\prob}{\addtocounter{problem}{1}\noindent {\bf Problem \theproblem.} ~}

\begin{document}
\header{1}

\textbf{Instructions.} You make work in groups, but you must write solutions yourself.
List collaborators on your submission. 

If you are asked to design an algorithm, please provide: (a) the pseudocode or precise description in words of the algorithm, (b) an explanation of the intuition for the algorithm, (c) a proof of correctness, (d) the running time of your algorithm and (e) justification for your running time analysis. 

\textbf{Submissions.} Please submit a PDF file. You may submit a scanned handwritten document, but a typed submission is preferred. Please assign pages to questions in Gradescope.

\begin{enumerate}[noitemsep,topsep=0pt]

% PROBLEM 1
\item \textbf{(15 points) Stable Matching Running Time.} 
  \begin{enumerate}
  \item 
  
  \item 
  \end{enumerate}

\vfill
    
    
% PROBLEM 2
\item \textbf{(20 points) Stable Matchings: K\&T Ch 1, Ex 5.}
  
  \begin{enumerate}
  \item \textbf{Strong Instability.} 
  \item \textbf{Weak Instability.} 
  \end{enumerate}
  
\vfill

% PROBLEM 3
\item \textbf{(15 points) Big-O.} 
  \begin{enumerate}
    \item $f(n) = \frac{1}{2}n^2$.
    \item $f(n) = n (\log n)^3$
    \item $f(n) =  \sum_{i=0}^{\lceil \log n\rceil} \frac{n}{2^i}$.
    \item $f(n) = \sum_{i=1}^n i^3$.
    \item $f(n) = 2^{(\log n)^2}$
  \end{enumerate}

% PROBLEM 4
\item \textbf{(20 points) Asymptotics. K\&T Ch 2, Ex 6.}

  \begin{enumerate}
  \item 
  \item 
  \item 
  \end{enumerate}


\medskip
% PROBLEM 5
\item \textbf{(10 points) DFS and BFS. K\&T Ch 3, Ex 5.} 


\bigskip
% PROBLEM 6
\item  \textbf{(20 points) Butterfly ID. K\&T Ch 3 Ex 4.}

\medskip
\item \textbf{(0 points).} How long did it take you to complete this assignment?

\end{enumerate}


\end{document}
